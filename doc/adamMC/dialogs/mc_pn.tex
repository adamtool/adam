usage: sh adam.sh mc_pn [-circ] [-cp] [-cr_abc] [-cr_com <arg>] [-cr_sys
       <arg>] [-d] [-enc <arg>] -f <LTL> -i <file> [-l <file>] [-max
       <arg>] [-noF] [-o <file>] [-p <abcParameters>] [-pnml] [-pre
       <process>] [-psst] [-stats] [-v] [-veri <verifier>]
Modelchecking 1-bounded Petri nets with inhibitor arcs against LTL.
 -circ,--circuit                      Saves the created circuit of the net
                                      as PDF. Attention: this could be
                                      really huge and dot could need lots
                                      of time!
 -cp,--check-precon                   Checks preconditions like 1-bounded.
                                      Takes some time, but should be used
                                      if you are not sure that your net
                                      fulfills  all necessary
                                      preconditions!
 -cr_abc,--red_abc                    Uses abc dfraig to reduce the
                                      circuit.
 -cr_com,--red_gates_com <arg>        Reduces the number of gates of the
                                      whole cirucit. That means it reduces
                                      the output of McHyper. Possible
                                      values: RX-G | RX-G-S | DS-G |
                                      DS-G-S | DS-G-S-EXTRA | NONE.
                                      Standard: NONE.
 -cr_sys,--red_gates_sys <arg>        Reduces the number of gates of the
                                      system's circuit. Possible values: G
                                      | G-EQCOM | G-I | G-I-EQCOM |
                                      G-I-EXTRA | G-I-EXTRA-EQCOM | NONE.
                                      Standard: NONE.
 -d,--debug                           Get some debug infos.
 -enc,--encoding <arg>                Encoding of the transitions in the
                                      circuit. Possible values: logEnc |
                                      expEnc. Standard: logEnc.
 -f,--formula <LTL>                   The formula which should be checked.
 -i,--input <file>                    The path to the input file.
 -l,--logger <file>                   The path to an optional logger file.
                                      If it's not set, the information
                                      will be send to the terminal.
 -max,--maximality <arg>              States which kind of maximality
                                      should be used. Possible values:
                                      IntC (interleaving calculated within
                                      the circuit)  | IntF (interleaving
                                      added to the formula)  | ConF
                                      (concurrent added to the formula)  |
                                      NONE. Standard: IntC.
 -noF,--noFile                        Does not write the formula to a file
                                      for giving it to MCHyper. This
                                      causes problems for huge formulas.
 -o,--output <file>                   The path to the output folder. If
                                      it's not given the path from the
                                      input file is used.
 -p,--abcParameters <abcParameters>   Parameters for the verifier /
                                      falsifier for abc. Standard: no
                                      parameters.
 -pnml                                Allows to read the Petri net from
                                      the PNML format rather than the
                                      standard format.
 -pre,--preProc <process>             Allows to excute any pre-process of
                                      abc before the actual veri- or
                                      falsifier is started.
 -psst,--silent                       Makes the tool voiceless.
 -stats,--statistics                  Calculates and prints some
                                      statistics for the call.
 -v,--verbose                         Makes the tool chatty.
 -veri,--verifier <verifier>          The set of ver- and falsifieres
                                      which should be executed in
                                      parallel.Note that even parallel
                                      execution has some overhead. Input
                                      format: v_1 | ... | v_n with v_i
                                      from {IC3, INT, BMC, BMC2, BMC3}.
                                      Standard: IC3
