usage: sh adam.sh mc_pnwt [-app <arg>] [-circ] [-cp] [-cr_abc] [-cr_com
       <arg>] [-cr_sys <arg>] [-d] [-enc <arg>] [-f <LTL | Flow-LTL
       formula>] -i <file> [-l <file>] [-max <arg>] [-noF] [-o <file>] [-p
       <abcParameters>] [-pre <process>] [-psst] [-st <arg>] [-stats] [-t]
       [-v] [-veri <verifier>]
Modelchecking Petri nets with transits against FlowLTL  or LTL.
 -app,--approach <arg>                   Chosing the sequential or
                                         parallel approach (with or
                                         without inhibitor arcs). Possible
                                         values: seq | seqIn | par |
                                         parIn. Standard: parIn.
 -circ,--circuit                         Saves the created circuit of the
                                         net as PDF. Attention: this could
                                         be really huge and dot could need
                                         lots of time!
 -cp,--check-precon                      Checks preconditions like
                                         1-bounded. Takes some time, but
                                         should be used if you are not
                                         sure that your net fulfills  all
                                         necessary preconditions!
 -cr_abc,--red_abc                       Uses abc dfraig to reduce the
                                         circuit.
 -cr_com,--red_gates_com <arg>           Reduces the number of gates of
                                         the whole cirucit. That means it
                                         reduces the output of McHyper.
                                         Possible values: RX-G | RX-G-S |
                                         DS-G | DS-G-S | DS-G-S-EXTRA |
                                         NONE. Standard: NONE.
 -cr_sys,--red_gates_sys <arg>           Reduces the number of gates of
                                         the system's circuit. Possible
                                         values: G | G-EQCOM | G-I |
                                         G-I-EQCOM | G-I-EXTRA |
                                         G-I-EXTRA-EQCOM | NONE. Standard:
                                         NONE.
 -d,--debug                              Get some debug infos.
 -enc,--encoding <arg>                   Encoding of the transitions in
                                         the circuit. Possible values:
                                         logEnc | expEnc. Standard:
                                         logEnc.
 -f,--formula <LTL | Flow-LTL formula>   The formula, either Flow-LTL or
                                         LTL, which should be checked.
 -i,--input <file>                       The path to the input file.
 -l,--logger <file>                      The path to an optional logger
                                         file. If it's not set, the
                                         information will be send to the
                                         terminal.
 -max,--maximality <arg>                 States which kind of maximality
                                         should be used. Possible values:
                                         IntC (interleaving calculated
                                         within the circuit)  | IntF
                                         (interleaving added to the
                                         formula)  | ConF (concurrent
                                         added to the formula)  | NONE.
                                         Standard: IntC.
 -noF,--noFile                           Does not write the formula to a
                                         file for giving it to MCHyper.
                                         This causes problems for huge
                                         formulas.
 -o,--output <file>                      The path to the output folder. If
                                         it's not given the path from the
                                         input file is used.
 -p,--abcParameters <abcParameters>      Parameters for the verifier /
                                         falsifier for abc. Standard: no
                                         parameters.
 -pre,--preProc <process>                Allows to excute any pre-process
                                         of abc before the actual veri- or
                                         falsifier is started.
 -psst,--silent                          Makes the tool voiceless.
 -st,--stuck <arg>                       Different formulas for the
                                         sequential approach to prevent
                                         runs from stucking in a subnet.
                                         Possible values: GFo | GFANDNpi |
                                         ANDGFNpi | GFoANDNpi. Standard:
                                         GFANDNpi.
 -stats,--statistics                     Calculates and prints some
                                         statistics for the call.
 -t,--trans                              Saves the transformed net in APT
                                         format and, in the case that dot
                                         is executable, as PDF.
 -v,--verbose                            Makes the tool chatty.
 -veri,--verifier <verifier>             The set of ver- and falsifieres
                                         which should be executed in
                                         parallel.Note that even parallel
                                         execution has some overhead.
                                         Input format: v_1 | ... | v_n
                                         with v_i from {IC3, INT, BMC,
                                         BMC2, BMC3}. Standard: IC3
