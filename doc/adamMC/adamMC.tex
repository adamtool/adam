\documentclass[runningheads]{./llncs/llncs}
% make a proper TOC despite llncs
\setcounter{tocdepth}{2}
\makeatletter
\renewcommand*\l@author[2]{}
\renewcommand*\l@title[2]{}
\makeatletter
\title{\tool: The User Guide}
\author{Manuel Gieseking\inst{1}  and Jesko Hecking-Harbusch\inst{2}}
\institute{
Carl von Ossietzky University of Oldenburg, Oldenburg, Germany\\
\email{manuel.gieseking@informatik.uni-oldenburg.de}
\and
Saarland University, Saarbr\"ucken, Germany\\
\email{hecking-harbusch@react.uni-saarland.de}
\\
\vspace{5mm}
\today}
\usepackage[toc,page]{appendix}

\usepackage{breakurl}
\usepackage[utf8]{inputenc}
\usepackage[english]{babel}
\usepackage{color}
\usepackage{xcolor}

\usepackage{amsmath}
\usepackage{amssymb}


% TikZ!
\usepackage{tikz}
\usetikzlibrary{arrows,shapes,automata, petri,positioning}
% Die default-Pfeilspitzen werden in PDF-Viewern erst bei ziemlichem Zoom
% angezeigt! => Andere Pfeilspitzen benutzen, damit man nicht st"andig denkt,
% dass Kanten ungerichtet seien.
\tikzset{>=latex'}

% Petri-Netze in bunt
\tikzstyle{place}=[circle,thick,draw=blue!75,fill=blue!20,minimum size=6mm]
\tikzstyle{transition}=[rectangle,thick,draw=black!75, fill=black!20,minimum size=4mm]
\tikzstyle{zustand}=[circle,fill=black]

\tikzstyle{envplace}=[circle,thick,draw=black!75,minimum size=6mm]
\tikzstyle{sysplace}=[circle,thick,draw=black!75,fill=black!20,minimum size=6mm]
\tikzstyle{transition}=[rectangle,thick,draw=blue!75,fill=blue!20,minimum size=6mm]


%Umgebung für Code-Fragmente
\usepackage{listings} \lstset{	
	morecomment=[s][\color{darkred}]{"}{"},
	escapeinside={\%*}{*)},
	commentstyle=\color{cdc_Green},
	keywordstyle=\color{blue},
	frame=single,
	%backgroundcolor=\color{lstColor},
%	breaklines=true,
	tabsize=4,
basicstyle=\scriptsize
%  showstringspaces=true,
%  xleftmargin=13pt,
%  xrightmargin = 5pt
}


\lstdefinelanguage{apt-format}
{
	morekeywords=[1]{.name, .type, .description, .transitions, .places, .flows, .initial_marking, .final_markings,
			.states, .labels, .arcs},
	alsoletter={.},
	keywordstyle=[1]\color{cdc_BlueM},
	morecomment=[s][\color{darkred}]{"}{"},
	morecomment=[s][\color{cdc_Green}]{/*}{*/},
	morecomment=[l][\color{cdc_Green}]{//},
	escapeinside={\%*}{*)},
	frame=single,
	%backgroundcolor=\color{lstColor},
	breaklines=true,
	xleftmargin=5pt,
	xrightmargin = 5pt,
	numbers=none,
}

\lstdefinelanguage{ebnf}
{
	morecomment=[s][\color{darkred}]{'}{'},
	escapeinside={\%*}{*)},
	morecomment=[s][\color{cdc_Green}]{(*}{*)},
	morecomment=[s][\color{cdc_BlueM}]{?}{?},
	frame=single,
	%backgroundcolor=\color{lstColor},
	breaklines=true,
	xleftmargin=5pt,
	xrightmargin = 5pt,
	showspaces=false,
	numbers=none  
}


\definecolor{darkred}{rgb}{0.601,0.001,0.001} %% dunkles Rot
\definecolor{cdc_BlueM}{rgb}{0.25,0.473,0.722} %% Heading Blue Medium (Oldenburg CD Ultramarine 60%)
\definecolor{cdc_Green}{rgb}{0.390,0.695,0.285} %% Heading Green (Oldenburg CD Chartreuse)


\newcommand{\tool}{{\sc Adam}}


\usepackage[
	plainpages=false,
	pdfpagelabels,
	breaklinks=true,
	a4paper,
	bookmarks,
	bookmarksopen=true,
	bookmarksnumbered=true,
	pdfauthor={Manuel Gieseking},
	pdftitle={ADAM: The User Guide},
]{hyperref}

%\usepackage{minitoc}
%\makeatletter
%\newcommand\tableofcontentsasdf{%
%    \@starttoc{toc}%
%}
%\makeatother

\def\tableofcontentsmine{
 \def\authcount##1{\setcounter{auco}{##1}\setcounter{@auth}{1}}
 \def\lastand{\ifnum\value{auco}=2\relax
                 \unskip{} \andname\
              \else
                 \unskip \lastandname\
              \fi}%
 \def\and{\stepcounter{@auth}\relax
          \ifnum\value{@auth}=\value{auco}%
             \lastand
          \else
             \unskip,
          \fi}%
 \@starttoc{toc}\if@restonecol\twocolumn\fi}

\usepackage{import}

\begin{document}
\maketitle
\begin{abstract}
\tool{} ({\bf{}A}nalyzer of {\bf{}D}istributed {\bf{}A}synchronous {\bf{}M}odels) is a synthesis tool for reactive
systems with multiple independent processes. The systems are modelled as Petri games,
games with one environment player and an arbitrary but bounded number of system players 
described as Petri nets. \tool{} synthesis winning strategies of the Petri games by reducing
the Petri game to a finite-graph game, solving  the graph game and reconstructing the
Petri game strategy from the one of the finite-graph game.

\tool{} is written in Java and uses \href{http://javabdd.sourceforge.net/index.html}{JavaBDD}
 as library for manipulating BDDs. The APT format
is used as input / output format and \href{https://github.com/CvO-Theory/apt}{APT}
 itself is used for parsing / rendering the Petri
games and for providing a data structure for Petri nets. For visualizing the finite-graph
games, the Petri games, and their strategies, \tool{} uses the \href{http://www.graphviz.org/}{DOT} format of Graphviz. 
\end{abstract}

\section*{Table of Contents}
\tableofcontentsmine

\section{Dependencies}
The dependencies for using \tool{} in the most comfortable way are: 

\begin{itemize}
	\item Java in a version greater or equal to 7 is needed.
 	\item For saving the games and strategies as a pdf file, dot (Graphviz) has to be installed
   in a version $\geq$ 2.36.0.
 	\item Please stick to the documentation of \href{http://javabdd.sourceforge.net/index.html}{JavaBDD},
   if you would like to use another library (like \href{http://www.itu.dk/research/buddy/}{BuDDy}, \href{http://vlsi.colorado.edu/~fabio/CUDD/cuddIntro.html}{CUDD}, \href{http://embedded.eecs.berkeley.edu/Research/cal_bdd/}{CAL}, etc.) for the BDD
   manipulation than the Java implementation of JavaBDD. The compiled file of such a
   library has then to be placed on the same level as the jar file of \tool{}. The parameter
   for choosing a different library is applicable from the user interface. By default,
   \tool{} uses BuDDy, if it's accessible, otherwise \tool{} falls back to the Java
   implementation of JavaBDD.
\end{itemize}

\section{Installation}
In order to install \tool{}, just extract the tarball, which can be found \href{http://www.uni-oldenburg.de/csd/adam}{here}. This should create a folder named 'adam' containing the program 'adam.jar', a compiled version of the BDD library BuDDy
'libbuddy.so', a README file on how to use \tool{} and a folder 'examples' containing some Petri games and
their strategies.

To run \tool{} on Linux systems, you can execute the bash script named 'adam.sh', also
placed in this folder, or directly use Java. More details on starting and using \tool{}, see
Sec.~\ref{sec:exModules}.

\section{Usage}
\label{sec:usage}
\tool{} has three categories of modules. There are converters from Petri games,
defined in the APT file format, to the .dot format or to a pdf document visualized by
Graphviz. Then, there are generators for some example Petri games, which
are also used within the benchmark suite. Finally, there are modules creating finite 
graph or Petri game strategies.

\subsection{List of Available Modules}
\label{sec:modules}
This section lists all programs \tool{} provides, which can also be printed by executing  
         'sh adam.sh' or 'java -jar adam.jar'.

%\begin{figure}[h!]
\lstinputlisting[mathescape]{dialogs/ui.tex}
%\caption{List of Modules.}
%\label{fig_comp_comput_inp}
%\end{figure}

A module can be executed by typing: 'sh adam.sh \textless{}module\textgreater' or 'java -jar adam.jar \textless{}module\textgreater'.
This prints a help dialogue how to use this module. All help dialogues can be found in the \hyperref[sec:appendix]{Appendix}.

\subsection{Executing the Modules}
\label{sec:exModules}
Executing 
\begin{lstlisting}
sh adam.sh
\end{lstlisting}
or 
\begin{lstlisting}
java -jar adam.jar
\end{lstlisting}
prints a list of all possible modules (see Sec~\ref{sec:modules}). The modules themselves can be started by executing
\begin{lstlisting}
sh adam.sh %*\textless{}*)module%*\textgreater*)
\end{lstlisting}
This prints a help dialogue stating the usage of this module and the available and 
necessary parameters. For each module this dialogue can be found in the \hyperref[sec:appendix]{Appendix}.

Subsequently, we give some standard calls for \tool{}:
 
\begin{lstlisting}
sh adam.sh pg2pdf -i ./folder/name.apt
\end{lstlisting}
This call saves the visualisation of the in APT format given Petri game as pdf file in the same folder.

\begin{lstlisting}
sh adam.sh win_strat -i ./folder/name.apt
\end{lstlisting}
This is a standard call for creating the winning strategies (finite graph and Petri game) if existent. It saves them (.dot, .pdf) in the same folder as the in APT format given Petri game. Also saves the visualization of the input Petri game with the used distribution of the places, visualized by different colouring of the places.

\begin{lstlisting}
sh adam.sh gen_workflow2 -m 2 -w 4 -o cw24
\end{lstlisting}
This call generates the Concurrent Machines (CM) examples of the \tool{} paper for two machines and four orders. It saves the resulting Petri game as APT, dot and pdf file to the given file name (cw24.apt, cw24.dot, cw24.pdf). 

Most of the optional parameters are self-explanatory. Thus, we only want to list some
special ones.

The parameter \emph{skip} forces \tool{} to skip all tests regarding the wellformedness of the 
Petri game in the input file. That is, it skips the test checking if the underlying 
Petri net is safe. Other tests, like checking if there is only one environment token, are 
not yet implemented. The tests of the wellformedness of the Petri net aren't skipped.

The parameter \emph{exp} is a flag, which forces \tool{} to use an experimental algorithm. In
Section 3.3. we explain the input format of the Petri games which \tool{} can use. There is 
stated that we need to distribute the places of the Petri game in disjunct sets 
satisfying some properties. This experimental version is much slower, but has the ability
to calculate the strategies for a subgroup of underlying Petri nets
(concurrency-preserving), without the need of any distribution of the places. 

The parameters starting with \emph{lib} concern the BDD library. With them you can exchange the
library used for the BDD calculation, if you have compiled C libs in the same folder as
\tool{}, and set some parameters for the node table and its usage.

\subsection{Creating your own input files}
The data format for modelling a Petri game is described \href{http://www.uni-oldenburg.de/fileadmin/user\_upload/f2inform-csd/adam/format.pdf}{here} in detail. There is also a formal grammar for parsing Petri nets given within the
document. In the following, we give a quick summary.

\tool{} works on safe Petri nets. If your net is not safe, you should transform it into a 
safe net before running \tool{} by adding additional places. Another precondition is that you
can use an arbitrary (bounded) number of system players, but only one environment player. 
Thus, you have to make sure that there is no reachable marking in which two environment 
places are marked at the same time. Furthermore, an environment player cannot convert into
a system player or vice versa. Thus, make sure that no environment token can occupy a 
system place or vice versa.

An input file contains of sections for places (\emph{.places}), transitions (\emph{.transitions}) 
and the connections between them (\emph{.flows}). You have to name the Petri game
(\emph{.name "my name"}) and set its type (\emph{.type LNP}). If you like to, you can give a 
description of the game with \emph{.description "lorem"}. The section \emph{.initial\_marking}
contains the initial marking of the Petri game.

Please do not use any underscores ('\_') within the names of your places while creating your
own input file in the APT file format. This causes trouble during calculating the Petri
game strategy.

For typing a place as an environment place annotate him with \emph{[env="true"]}, all other
places are automatically typed as belonging to the system players. Do the same for 
bad places with \emph{[bad="true"]}.

The places of the Petri game aren't just divided into environment and system places, but
also into groups of places stating these are the places a given token can lie on. Thus, 
you can annotate the places with the \emph{[token=\textless{}number\textgreater]} key-value pair, where \textless{}number\textgreater is 
the number of the group this place belongs to. You only have to annotate the system 
places. Environment places are automatically marked as group 0. Thus, the numbers
annotated must start from 1 and it is not allowed to omit some natural number between 1
and the maximum number annotated. Be aware that the partition has to be disjunct in the 
sense that no two places in one group can be marked at the same time. \tool{} has a feature 
to support you by annotating the places or even completely annotate the places on its own.
To use this, just don't annotate any places with \emph{[token=\textless{}number\textgreater{}]} and use, for example, 
the \emph{win\_strat} module. If \tool{} can annotate the places on its own, you can read the
annotation by the colouring of the places in the resulting pdf file of the Petri game.
Otherwise, \tool{} prints some invariants which should support you by following the token 
through the net.

Here is an example input file:

\begin{lstlisting}[language=apt-format]
.name "my name"
.type LNP

.places
env1[env="true"]
bad0[token=1, bad="true"]
bad1[token=2, bad="true"]
good[token=2]
...
env0[env="true"]

.transitions
t1 t2
t3
...

.flows
t1: {env1} -> {env0}
t1: {bad0, bad1} -> {bad0}
...

.initial_marking {env1, good}
\end{lstlisting}

For more examples, see the examples folder within the tarball (\emph{./examples/}) or 
the \href{http://www.uni-oldenburg.de/fileadmin/user\_upload/f2inform-csd/adam/format.pdf}{file} defining the input format.

\section{Format}
This document describes the file format of \tool{}, a tool for the automatic synthesis of distributed systems with multiple concurrent processes. For further information about the tool feel free to visit \url{http://www.uni-oldenburg.de/csd/adam}.

\tool{} uses the data structure for Petri nets and the parser provided by APT\footnote{\url{https://github.com/CvO-Theory/apt}}. Therefore, we rehash some information about the grammar of the parser and the special case how to define Petri games in the APT file format.

In general a file in the APT file format is structured in sections labeled with \textit{.section} with its content added below or directly behind this keyword. For the usability of writing the document, whitespaces are ignored.

The name of the file or Petri game is given by 

\begin{lstlisting}[language=apt-format]
.name "name"
\end{lstlisting}

Every character is allowed to be used within the string, apart from the doubled quotation marks, line breaks and tabulators.

Since the file format itself is more general and also used for transition systems in other contexts, we have to specify the type of the graph. This is done by

\begin{lstlisting}[language=apt-format]
.type LPN
\end{lstlisting}

Optionally we can add a description to the given net with

\begin{lstlisting}[language=apt-format]
.description "Some describing text of the Petri game"
\end{lstlisting}

This string is the only one which -- different to the string mentioned above -- can additionally use line breaks for defining longer descriptions.

The places are given by

\begin{lstlisting}[language=apt-format]
.places
env1[env="true"]
bad0[token=2, bad="true"]
bad1[token=3, bad="true"]
...
env0[env="true"]
\end{lstlisting}

For typing a place as an environment place annotate him with [\texttt{env="true"}], all other places are typed as belonging to the system players. Do the same for bad places with [\texttt{bad="true"}].

The places of the Petri game aren't just divided into environment and system places, but
also into groups of places stating these are the places a given token can lie on. Thus, 
you can annotate the places with the [\texttt{token=\textless{}number\textgreater}] key-value pair, where \textless{}number\textgreater is the number of the group this place belongs to. You only have to annotate the system 
places. Environment places are automatically marked as group 0. Thus, the numbers
annotated must start from 1 and it is not allowed to omit some natural number between 1
and the maximum number annotated. Be aware that the partition has to be disjunct in the 
sense that no two places in one group can be marked at the same time. \tool{} has a feature 
to support you by annotating the places or even completely annotate the places on its own.
To use this, just don't annotate any places with [\texttt{token=\textless{}number\textgreater{}}] and use, for example, 
the \emph{win\_strat} module. If \tool{} can annotate the places on its own, you can read the
annotation by the colouring of the places in the resulting pdf file of the Petri game.
Otherwise, \tool{} prints some invariants which should support you by following the token 
through the net.

The identifiers of the places (like \texttt{env1}, \texttt{bad0}, etc.) are allowed to be integers or strings beginning with a letter and containing only characters (a-z, A-Z) and integers. Different to the grammar described in Listing~\ref{lst:grammar}, there are no underscores valid within the identifier of the places for Petri games. This would cause some trouble while calculating the strategy for the Petri game.

For defining the transitions of the Petri game the same rules for the identifiers, apart from the underscores, are implemented. That is, additionally the identifiers can start with an underscore and can contain some within.

\begin{lstlisting}[language=apt-format]
.transitions
t1
t2
...
\end{lstlisting}

To define the connection between places and transitions the section \textit{.flows} is used. This is done by listing the pre- and postsets of every transition separately.

\begin{lstlisting}[language=apt-format]
.flows
t1: {s1} -> {s1}
\end{lstlisting}

Thus, here the place \texttt{s1} has a self loop with transition \texttt{t1}. The preset is intuitively defined in front of the arrow and the postset behind.

At the moment the Petri games in \tool{} are restricted to safe Petri nets. Thus, there is no need for multi-weight edges. But since this restriction should not last for ever, we are already describing the not $1$-bound case here. 

Multi-weight edges are defined by a multiplier given by a natural number stating how many tokens should be used or transmitted respectively.

\begin{lstlisting}[language=apt-format]
.flows
t1: {} -> {s2, 2*s1, s2, 0*s3, 3*s2} 
\end{lstlisting}

The edge weight $1$ need not to be given as a multiplier additionally. The multiplier $0$ -- an edge with weight zero -- is the same as not listing the place at all. The empty pre- or postset can be defined by an empty set $\{\}$. Furthermore, the place \texttt{s2} is listed three times in the postset of \texttt{t1}. Twice with the invisible multiplier $1$ and once with the multiplier $3$. This yields in an edge with weight $5$ in the resulting Petri net. Remark that the multiplier is not commutative, since the identifiers of the places can also -- as mentioned above -- be only integers and thus the semantics would not be unique any more.  

The initial marking of the Petri net is given analogously to the pre- and postsets as multisets with multipliers:

\begin{lstlisting}[language=apt-format]
.initial_marking {2*s1, s2}
\end{lstlisting}

Thus, initially there are two token on place \texttt{s1} and one on \texttt{s2}. Is this section omitted, the resulting Petri net won't have any token.

The last property is the ability to add comments at any place in the file. Therefore we use the comments notations of languages like C. Thus, a comment influencing the whole line is initialized by \textit{//}. A whole area of a comment can be framed by \textit{/* */}, which is also possible to contain multiple lines.

\begin{lstlisting}[language=apt-format]
// comment
.flows
t1: {} -> {2*s1,/*s2,*/ 3*s3}
\end{lstlisting}

All in all, a complete example of a Petri game defined in the APT format can look like:

%All in all, a complete example of a Petri game visualized in Figure~\ref{img_df_pn} is given in the APT format in Listing~\ref{lst_df_pn}.

\begin{lstlisting}[literate=, captionpos=b, caption=Example of a Petri game, label = lst_df_pn,language=apt-format]
// The environment decides between A or B, and the system players should imitate this by choosing A' or B'. http://www.react.uni-saarland.de/publications/FO14.pdf
.name "SameDecision"
.type LPN
.description "Testing the same decision."

.places
Env[env="true"]
A[env="true"]
B[env="true"]
Sys[token=1]
A_[token=1]
B_[token=1]
EA[env="true"]
EB[env="true"]
qbad[token=1, bad="true"]

.transitions
t1 t2
test1 test2
t1_ t2_
tbad1 tbad2 tbad3 tbad4

.flows
t1: {Env} -> {A}
t2: {Env} -> {B}
test1: {A,Sys} -> {Sys,EA}
test2: {B,Sys} -> {Sys,EB}
t1_: {Sys} -> {A_}
t2_: {Sys} -> {B_}
// bad states
tbad1: {A_,EB} -> {EB,qbad}
tbad2: {B_,EA} -> {EA,qbad}
tbad3: {A_,B} -> {B,qbad}
tbad4: {B_,A} -> {A,qbad}

.initial_marking {Env,Sys}
\end{lstlisting}

%\begin{minipage}{\textwidth}
%\centering
%\begin{tikzpicture}[node distance=2cm,>=stealth',bend angle=45,auto]
%	\node [envplace,tokens=1] (env)  [label=above:$\text{Env}$]                                  {};
%	\node [sysplace] (sys) [below of=env,label=above:$\text{Sys}$]                     {};
%	\node[token] at (sys) {};
%	\node [transition] (test1)  [left of=sys,label=above:$\text{test}_1$] {};
%	\node [transition] (test2)  [right of=sys,label=above:$\text{test}_2$] {};
%	\node [envplace] (a)  [left of=test1,label=left:$A$]                                  {};
%	\node [envplace] (b)  [right of=test2,label=right:$B$]                                  {};
%	\node [transition] (t1)  [above of=a,label=above:$t_1$] {};
%	\node [transition] (t2)  [above of=b,label=above:$t_2$] {};
%	\node [transition] (t1_)  [below of=test1,label=left:$t_1'$,xshift=1cm] {};
%	\node [transition] (t2_)  [below of=test2,label=right:$t_2'$,xshift=-1cm] {};
%	\node [sysplace] (a_) [below of=t1_,label=left:$A'$]                     {};
%	\node [sysplace] (b_) [below of=t2_,label=right:$B'$]                     {};
%	\node [envplace] (ea)  [below of=a,label=left:$EA$]                                  {};
%	\node [envplace] (eb)  [below of=b,label=right:$EB$]                                  {};
%	\node [transition] (tb)  [below of=ea,label=left:$t_{B\bot}$] {};
%	\node [transition] (ta)  [below of=eb,label=right:$t_{A\bot}$] {};
%	\node [sysplace,accepting] (bot) [below of=a_,right of=a_,label=below:$\bot$,xshift=-1cm]                     {};

%	\path[->] 	(t1)  		edge [pre]                            (env)
%					edge [post]                            (a)
%		 	(t2)  		edge [pre]                            (env)
%					edge [post]                            (b)
%		 	(test1) 	edge [pre]                            (a)
%					edge [pre,bend left]                            (sys)
%					edge [post,bend right]                            (sys)
%					edge [post]                            (ea)
%		 	(test2) 	edge [pre]                            (b)
%					edge [pre,bend right]                            (sys)
%					edge [post,bend left]                            (sys)
%					edge [post]                            (eb)
%		 	(t1_)  		edge [pre]                            (sys)
%					edge [post]                            (a_)
%		 	(t2_)  		edge [pre]                            (sys)
%					edge [post]                            (b_);
%	\path[->]	(tb) 		edge [pre, bend right]                            (b_)
%					edge [pre, bend right]                            (ea)
%					edge [post, bend left]                            (ea)
%					edge [post,bend right]                            (bot);
%	\path[->]		 	(ta) 		edge [pre, bend left]                            (a_)
%					edge [pre, bend right]                            (eb)
%					edge [post, bend left]                            (eb)
%					edge [post, bend left]                            (bot);
%\end{tikzpicture}
%%\caption{With Listing \ref{lst_df_pn} defined Petri game}
%\label{img_df_pn}
%\end{minipage}

%Eine Grammatik wie diese einzelnen Definitionen in dem implementierten Parser umgesetzt werden, befindet sich in Listing \ref{grammatik_petri_netze}.
%\abs
%Die einzelnen Sektionen d"urfen in beliebiger Reihenfolge aufgef"uhrt werden, wobei die Abschnitte \textit{.name}, \textit{.initial\_marking} und \textit{.description} null- oder genau einmal auftreten d"urfen und die Abschnitte \textit{.places}, \textit{.transitions} und \textit{.flows} nullmal oder beliebig h"aufig verwandt werden d"urfen. Wenn sie mehrfach auftreten, dann werden alle Werte f"ur die Definition des Netzes verwandt. Die Sektion \textit{.type} muss genau einmal aufgef"uhrt sein.

%\subsubsection{Grammatik f"ur Petri-Netze mit und ohne Beschriftungen}
%\label{sec_parser_lpn_grammatik}
%In Listing \ref{grammatik_petri_netze} ist eine Grammatik in EBNF mit dem Startsymbol \texttt{S} angegeben, welche die korrekte Syntax von Petri-Netzen mit und ohne Beschriftungen im APT-Format definiert. Dabei ist zu beachten, dass die Besonderheit des Formates bez"uglich der H"aufigkeit des Auftretens der Sektionen nicht in der Grammatik "ubersichtlich dargestellt werden konnte und auch im implementierten Parser mithilfe von Algorithmen beim Parsevorgang gel"ost ist. Ebenso ist der konkrete Umgang mit den Whitespaces wie auch den Kommentaren in der Grammatik nicht direkt dargestellt; diese werden zwar gelesen, aber f"ur die weitere Verarbeitung verworfen.
%\abs

%Wichtig ist also zu beachten, dass die Reihenfolgen der einzelnen Sektionen zwar beliebig ist, jedoch k"onnen die Abschnitte f"ur Stellen, Transitionen, Kanten und der Endmarkierung beliebig h"aufig angegeben werden. Die Sektionen f"ur Namen und Beschreibung sind optional, wobei die Definition f"ur den Typ genau einmal auftauchen muss. Die spezielle Sequenz \texttt{EOF} besagt, dass nach den Sektionen die Datei auch zu Ende sein soll. Dies ist f"ur aussagekr"aftigere Fehlermeldungen hilfreich.

The whole grammar which the parser accepts is given by 
\begin{lstlisting}[captionpos=b, caption=Grammar of the APT format for labelled Petri nets, label = lst:grammar,language=ebnf]
S               = { name | type | description | places 
                  | transitions | flows 
                  | initial_marking | final_markings },
                  ? EOF ?;
name            = '.name', STR;
type            = '.type', ('LPN' | 'PN');
description     = '.description', (STR | STR_MULTI);
places          = '.places', place;
place           =  | idi, [opts], place;
opts            = '[', option, ']';
option          = ID, '=', STR, [',', option];
transitions     = '.transitions', transition;
transition      =  | idi, [opts], transition;
flows           = '.flows', flow;
flow            = | idi, ':', set, '->', set, flow;
set             = '{', [objs], '}';
objs            = obj, [',', objs]; 
obj             = (INT, '*', idi) | idi;
initial_marking = '.initial_marking', [set];
final_markings  = '.final_markings', final_marking;
final_marking   = | set, final_marking;
idi             = ID | INT;
INT             = DIGIT, {DIGIT};
ID              = (CHAR | '_'), {CHAR | DIGIT | '_'};
DIGIT           = '0' | '1' | '2' | '3' | '4' | '5' 
                | '6' | '7' | '8' | '9';
CHAR            = 'A' | 'B' | 'C' | 'D' | 'E' | 'F'
                | 'G' | 'H' | 'I' | 'J' | 'K' | 'L'
                | 'M' | 'N' | 'O' | 'P' | 'Q' | 'R'
                | 'S' | 'T' | 'U' | 'V' | 'W' | 'X'
                | 'Y' | 'Z' |
                  'a' | 'b' | 'c' | 'd' | 'e' | 'f'
                | 'g' | 'h' | 'i' | 'j' | 'k' | 'l'
                | 'm' | 'n' | 'o' | 'p' | 'q' | 'r'
                | 's' | 't' | 'u' | 'v' | 'w' | 'x'
                | 'y' | 'z';
STR             = '"', {ALLCHAR - ('"' | '\n' | '\r' | '\t')}, '"';
STR_MULTI       = '"', {ALLCHAR - ('"' | '\t')}, '"';
COMMENT         = '//', {ALLCHAR - ('\n'|'\r')}, ['\r'], '\n';
	        | '/*', {ALLCHAR} - '*/', '*/';
WHITESPACE      = ' ' | '\n' | '\r' | '\t';
ALLCHAR         = ? All possible symbols ?;
\end{lstlisting}
%\label{lst:grammar}


\section{Contact}
We appreciate your feedback on \tool{}! Please send any bugs, comments, or questions to: \emph{manuel.gieseking(at)informatik.uni-oldenburg.de}

Thank you for using \tool{}!

\appendix
\section{Appendix}
\label{sec:appendix}
Following the help dialogues of each module. That is, how to call the module, the possible 
and needed parameters including their explanations.
\begin{subappendices}
\subimport{./dialogs/}{helpDialogs}
\end{subappendices}

\end{document}
